\chapter{Gestión de Transacciones}

\section{Conceptos Básicos}

Una \textbf{transacción} es un asiento contable que registra un hecho económico siguiendo el principio de \textbf{partida doble}: el total de los débitos debe ser igual al total de los créditos.

\subsection{Componentes de una Transacción}

\begin{itemize}
    \item \textbf{Fecha}: Cuando ocurrió la operación
    \item \textbf{Leyenda}: Descripción del hecho económico
    \item \textbf{Entradas}: Líneas del asiento con cuentas, importes y clasificación (Debe/Haber)
\end{itemize}

\section{Crear una Transacción}

\begin{enumerate}
    \item Acceda a \textbf{Transacciones > Crear Transacción}
    \item Complete los datos generales:
    \begin{itemize}
        \item Fecha de la transacción
        \item Leyenda descriptiva
    \end{itemize}
    \item Agregue las entradas del asiento:
    \begin{itemize}
        \item Haga clic en \textbf{Agregar Entrada}
        \item Seleccione la cuenta
        \item Ingrese el importe
        \item Especifique Debe o Haber
    \end{itemize}
    \item Verifique el balance (Total Debe = Total Haber)
    \item Guarde la transacción
\end{enumerate}

\begin{figure}[H]
    \centering
    \includegraphics[width=0.9\textwidth]{img/crear_transaccion.png}
    \caption{Formulario de creación de transacción}
    \label{fig:crear_trans}
\end{figure}

\begin{warningbox}
El sistema solo permitirá guardar cuando el asiento esté balanceado (Debe = Haber).
\end{warningbox}

\section{Estados de las Transacciones}

Las transacciones pueden tener tres estados:

\begin{description}
    \item[Por Validar] Estado inicial. Puede editarse y eliminarse.
    \item[Validada] Revisada y confirmada. Lista para libro diario. No puede editarse.
    \item[Cerrada] Incluida en un libro diario. No puede modificarse nunca.
\end{description}

\begin{figure}[H]
    \centering
    \includegraphics[width=0.8\textwidth]{img/estados_transaccion.png}
    \caption{Flujo de estados de una transacción}
    \label{fig:estados}
\end{figure}

\section{Validación de Transacciones}

La validación confirma que una transacción está correcta:

\begin{enumerate}
    \item Acceda a \textbf{Transacciones > Buscar Transacciones}
    \item Filtre por estado "Por Validar"
    \item Revise cada transacción:
    \begin{itemize}
        \item Fecha correcta
        \item Leyenda descriptiva
        \item Cuentas apropiadas
        \item Importes correctos
        \item Asiento balanceado
    \end{itemize}
    \item Haga clic en \textbf{Validar}
\end{enumerate}

\begin{tipbox}
Valide transacciones regularmente. Solo las transacciones validadas pueden incluirse en libros diarios.
\end{tipbox}

\section{Búsqueda de Transacciones}

\subsection{Búsqueda Rápida}

Use la barra superior para búsquedas simples por:
\begin{itemize}
    \item Número de transacción
    \item Palabras en la leyenda
\end{itemize}

\subsection{Búsqueda Avanzada}

Acceda a \textbf{Transacciones > Buscar} para filtrar por:

\begin{itemize}
    \item \textbf{Fecha}: Específica o rango
    \item \textbf{Estado}: Por Validar, Validada, Cerrada
    \item \textbf{Usuario}: Creador de la transacción
    \item \textbf{Cuenta}: Transacciones que incluyan una cuenta específica
    \item \textbf{Monto}: Rango de importes
    \item \textbf{Texto}: Búsqueda en leyendas
\end{itemize}

\begin{figure}[H]
    \centering
    \includegraphics[width=0.9\textwidth]{img/buscar_transacciones.png}
    \caption{Filtros de búsqueda avanzada}
    \label{fig:busqueda}
\end{figure}

Puede combinar múltiples filtros para búsquedas precisas.

\section{Plan de Cuentas}

El plan de cuentas es la estructura de clasificación contable.

\subsection{Visualizar el Plan de Cuentas}

Acceda a \textbf{Plan de Cuentas > Ver Plan de Cuentas} para consultar todas las cuentas disponibles.

\begin{figure}[H]
    \centering
    \includegraphics[width=0.9\textwidth]{img/plan_cuentas.png}
    \caption{Vista del plan de cuentas}
    \label{fig:plan_cuentas}
\end{figure}

\subsection{Crear Nuevas Cuentas}

Usuarios con permisos apropiados pueden agregar cuentas:

\begin{enumerate}
    \item Acceda a \textbf{Plan de Cuentas > Crear Cuenta}
    \item Complete los datos:
    \begin{itemize}
        \item Código único
        \item Nombre descriptivo
        \item Naturaleza (Deudora o Acreedora)
    \end{itemize}
    \item Guarde la cuenta
\end{enumerate}

\begin{notebox}
Las cuentas no pueden modificarse una vez creadas. Si comete un error, desactive la cuenta y cree una nueva.
\end{notebox}

\subsection{Desactivar Cuentas}

En lugar de eliminar, las cuentas se desactivan:
\begin{itemize}
    \item Estado \textbf{Activo}: Disponible para nuevas transacciones
    \item Estado \textbf{Inactivo}: No disponible, pero se mantiene por integridad histórica
\end{itemize}

\begin{warningbox}
Las cuentas utilizadas en transacciones no pueden eliminarse, solo desactivarse.
\end{warningbox}

\section{Ejemplo Práctico}

\textbf{Caso}: Registrar compra de mercadería en efectivo por \$10,000

\begin{enumerate}
    \item Crear nueva transacción con fecha actual
    \item Leyenda: "Compra de mercadería en efectivo"
    \item Agregar entradas:
    \begin{itemize}
        \item Cuenta "Mercaderías" - Debe \$10,000
        \item Cuenta "Caja" - Haber \$10,000
    \end{itemize}
    \item Verificar balance: Debe = Haber = \$10,000
    \item Guardar transacción (quedará en estado "Por Validar")
\end{enumerate}