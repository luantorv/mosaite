\chapter{Glosario de Términos}

\begin{description}[leftmargin=!,labelwidth=4cm,style=nextline]

\item[Asiento Contable] Registro de una operación económica que afecta al patrimonio. En Mosaite, sinónimo de transacción.

\item[Balance] Igualdad entre el total del Debe y el total del Haber en un asiento. Requisito fundamental de la partida doble.

\item[Cerrar] Acción de incluir una transacción en un libro diario, después de lo cual no puede modificarse.

\item[Cobertura] Porcentaje de transacciones validadas o cerradas sobre el total registrado. Indica calidad del seguimiento.

\item[Cuenta] Elemento del plan de cuentas que representa un tipo de activo, pasivo, patrimonio, ingreso o egreso.

\item[Debe] Lado izquierdo del asiento. Registra aumentos en activos y gastos, disminuciones en pasivos e ingresos.

\item[Entrada] Cada línea individual dentro de un asiento contable. Una transacción contiene múltiples entradas.

\item[Estado] Situación actual de una transacción: Por Validar, Validada o Cerrada.

\item[Haber] Lado derecho del asiento. Registra disminuciones en activos y gastos, aumentos en pasivos e ingresos.

\item[Leyenda] Descripción textual del hecho económico registrado en un asiento.

\item[Libro Diario] Registro cronológico de todas las operaciones contables de un período. En Mosaite se genera en PDF.

\item[Naturaleza] Característica de una cuenta que indica si es Deudora (aumenta con Debe) o Acreedora (aumenta con Haber).

\item[Partida Doble] Principio contable: toda operación tiene doble efecto. Lo que entra debe ser igual a lo que sale (Debe = Haber).

\item[Plan de Cuentas] Listado ordenado y codificado de todas las cuentas utilizadas en la contabilidad.

\item[RAG] Retrieval Augmented Generation. Técnica de IA que combina búsqueda de información con generación de respuestas.

\item[Rol] Conjunto de permisos asignados a un usuario que determina qué acciones puede realizar.

\item[Saldo] Diferencia entre el total del Debe y el total del Haber de una cuenta.

\item[Transacción] Registro completo de una operación contable. Sinónimo de asiento contable en Mosaite.

\item[Validar] Confirmar que una transacción está correcta y pasarla de "Por Validar" a "Validada".

\end{description}

\section*{Abreviaturas}

\begin{description}[leftmargin=!,labelwidth=4cm]

\item[IA] Inteligencia Artificial

\item[PDF] Portable Document Format

\item[RAG] Retrieval Augmented Generation

\item[SQL] Structured Query Language

\item[TUI] Terminal User Interface

\item[URL] Uniform Resource Locator

\end{description}