\chapter{Herramientas de Inteligencia Artificial}

\section{Chat RAG - Asistente Contable}

El Chat RAG es un asistente inteligente que responde preguntas sobre teoría contable y funcionamiento del sistema.

\subsection{¿Qué es RAG?}

RAG (Retrieval Augmented Generation) combina búsqueda de información con generación de respuestas, proporcionando respuestas contextualizadas basadas en documentación teórica.

\begin{figure}[H]
    \centering
    \includegraphics[width=0.9\textwidth]{img/chat_rag.png}
    \caption{Interfaz del asistente contable}
    \label{fig:chat}
\end{figure}

\subsection{Uso del Chat}

\begin{enumerate}
    \item Acceda a \textbf{Chat} desde el menú lateral
    \item Escriba su pregunta en lenguaje natural
    \item Presione Enter o haga clic en Enviar
    \item El asistente procesará y responderá
\end{enumerate}

\subsection{Tipos de Consultas}

El Chat puede responder sobre:

\begin{description}
    \item[Teoría Contable] Conceptos fundamentales, principios, definiciones
    \item[Plan de Cuentas] Información sobre cuentas predeterminadas
    \item[Uso del Sistema] Cómo realizar tareas en Mosaite
\end{description}

\textbf{Ejemplos de preguntas:}
\begin{itemize}
    \item "¿Qué es el principio de partida doble?"
    \item "¿Cuál es la diferencia entre activo y pasivo?"
    \item "¿Cómo creo una nueva transacción?"
    \item "¿Para qué se usa la cuenta Proveedores?"
\end{itemize}

\subsection{Cómo Hacer Buenas Preguntas}

\begin{tipbox}
\textbf{Consejos para mejores respuestas:}
\begin{itemize}
    \item Sea específico en su pregunta
    \item Proporcione contexto cuando sea necesario
    \item Haga una pregunta a la vez
    \item Use lenguaje natural y claro
\end{itemize}
\end{tipbox}

\subsection{Limitaciones del Chat}

\begin{warningbox}
El Chat RAG \textbf{NO} puede:
\begin{itemize}
    \item Acceder a sus transacciones específicas
    \item Realizar cálculos sobre sus datos
    \item Generar asientos automáticamente
    \item Proporcionar asesoramiento profesional legal
\end{itemize}
\end{warningbox}

Para consultas sobre sus datos contables, use ConsultorIA.

\section{ConsultorIA - Búsqueda por Lenguaje Natural}

ConsultorIA permite consultar sus datos contables usando lenguaje natural.

\begin{figure}[H]
    \centering
    \includegraphics[width=0.9\textwidth]{img/consultoria.png}
    \caption{Interfaz de ConsultorIA}
    \label{fig:consultoria}
\end{figure}

\subsection{¿Qué Hace ConsultorIA?}

\begin{itemize}
    \item Interpreta preguntas en lenguaje cotidiano
    \item Las convierte automáticamente en consultas SQL
    \item Busca en su base de datos de transacciones
    \item Presenta resultados comprensibles
\end{itemize}

\subsection{Uso de ConsultorIA}

\begin{enumerate}
    \item Acceda a \textbf{Transacciones > Búsqueda por Lenguaje Natural}
    \item Escriba su pregunta
    \item Haga clic en \textbf{Consultar}
    \item Revise los resultados
\end{enumerate}

\subsection{Tipos de Consultas}

\textbf{Sobre gastos o ingresos:}
\begin{itemize}
    \item "¿Cuánto gasté en proveedores en abril?"
    \item "¿Cuál fue el total de ventas este mes?"
\end{itemize}

\textbf{Sobre cuentas específicas:}
\begin{itemize}
    \item "¿Cuál es el saldo de la cuenta Caja?"
    \item "Muéstrame movimientos de Bancos"
\end{itemize}

\textbf{Consultas temporales:}
\begin{itemize}
    \item "¿Cuántas transacciones registré ayer?"
    \item "Muestra asientos de la semana pasada"
\end{itemize}

\textbf{Consultas comparativas:}
\begin{itemize}
    \item "¿Gasté más en marzo o abril?"
    \item "Compara ventas de enero con febrero"
\end{itemize}

\subsection{Interpretación de Resultados}

Los resultados pueden presentarse como:

\begin{description}
    \item[Valores numéricos] "El total gastado fue \$45,320"
    \item[Listas] Detalle de transacciones encontradas
    \item[Tablas comparativas] Comparaciones por período
    \item[Respuestas descriptivas] Resúmenes de actividad
\end{description}

\subsection{Diferencias Entre Chat y ConsultorIA}

\begin{table}[H]
\centering
\begin{tabular}{|p{4cm}|p{5cm}|p{5cm}|}
\hline
\textbf{Característica} & \textbf{Chat RAG} & \textbf{ConsultorIA} \\
\hline
Propósito & Teoría y uso del sistema & Consultas sobre sus datos \\
\hline
Acceso a datos & NO accede a transacciones & SÍ accede a sus transacciones \\
\hline
Tipo de respuestas & Conceptuales y educativas & Numéricas y basadas en datos \\
\hline
Ejemplo de uso & "¿Qué es un activo?" & "¿Cuánto gasté en mayo?" \\
\hline
\end{tabular}
\caption{Comparación entre Chat RAG y ConsultorIA}
\label{tab:comparacion}
\end{table}

\subsection{Consejos para Mejores Resultados}

\begin{enumerate}
    \item Sea claro y específico en sus preguntas
    \item Especifique períodos cuando sea relevante
    \item Use nombres correctos de cuentas
    \item Para consultas complejas, divídalas en varias simples
    \item Si no obtiene el resultado esperado, reformule
\end{enumerate}

\begin{notebox}
ConsultorIA solo consulta información, no modifica datos. Es una herramienta de solo lectura.
\end{notebox}