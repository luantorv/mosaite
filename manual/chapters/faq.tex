\chapter{Preguntas Frecuentes}

\section{Sobre el Sistema}

\subsection{¿Qué es Mosaite?}

Mosaite es una plataforma web educativa para la práctica de contabilidad básica. Permite registrar asientos contables, generar libros diarios y consultar información mediante herramientas de inteligencia artificial.

\subsection{¿Necesito instalar algo?}

No. Mosaite funciona completamente en el navegador web. Solo necesita la URL de acceso y conexión a la red local.

\subsection{¿Puedo acceder desde mi casa?}

Depende de la configuración. Por defecto funciona en red local. Consulte con su administrador para acceso remoto.

\subsection{¿Mis datos están seguros?}

Los datos se almacenan localmente en el servidor institucional. No se envía información a servicios externos. Sin embargo, no ingrese datos reales de empresas.

\section{Sobre el Uso}

\subsection{Olvidé mi contraseña}

Contacte a su profesor o administrador. Ellos pueden restablecerla. No hay recuperación automática.

\subsection{¿Puedo ver transacciones de otros estudiantes?}

No, a menos que sea profesor o administrador. Los estudiantes solo ven sus propios registros.

\subsection{¿El sistema me califica?}

No. Mosaite es una herramienta de práctica. Su profesor revisará y evaluará su trabajo.

\section{Sobre Transacciones}

\subsection{Cometí un error en una transacción}

Si está "Por Validar", puede editarla. Si ya fue validada, solo un administrador puede revertir la validación.

\subsection{¿Qué significa "cerrada"?}

Que está incluida en un libro diario. No puede editarse ni eliminarse para mantener integridad contable.

\subsection{Mi asiento no balancea}

Verifique que Total Debe = Total Haber. El sistema no permite guardar hasta que estén balanceados.

\subsection{¿Puedo eliminar una transacción?}

Sí, si está "Por Validar" y tiene permisos. Las validadas o cerradas no pueden eliminarse.

\section{Sobre el Plan de Cuentas}

\subsection{¿Puedo agregar cuentas?}

Sí, con permisos de Accountant o superior. Acceda a "Plan de Cuentas > Crear Cuenta".

\subsection{¿Puedo modificar una cuenta?}

No. Las cuentas no se modifican. Si hay error, desactive la incorrecta y cree una nueva.

\subsection{¿Puedo eliminar una cuenta?}

Solo desactivarla. Las cuentas usadas en transacciones no se eliminan completamente.

\section{Sobre Libros Diarios}

\subsection{¿Cuándo genero un libro diario?}

Al completar un período contable y tener transacciones validadas que desee formalizar.

\subsection{¿Puedo modificar un libro generado?}

No. Los libros son documentos finales. Si necesita cambios, debe eliminarlo (solo Admin) y regenerarlo.

\section{Sobre Herramientas de IA}

\subsection{¿Diferencia entre Chat y ConsultorIA?}

\textbf{Chat RAG}: Responde teoría contable y uso del sistema.

\textbf{ConsultorIA}: Consulta SUS datos contables específicos.

\subsection{¿El Chat crea asientos automáticamente?}

No. Es una herramienta de aprendizaje que responde preguntas, no crea transacciones.

\subsection{ConsultorIA no entiende mi pregunta}

Reformule de manera más simple y específica. Use nombres correctos de cuentas y especifique períodos claramente.

\section{Problemas Técnicos}

\subsection{La página está congelada}

Recargue la página (F5). Si persiste, cierre y reabra el navegador.

\subsection{Perdí mi trabajo}

Mosaite guarda al hacer clic en "Guardar". Si no guardó, se perderá. Guarde frecuentemente.

\subsection{El sistema está lento}

Puede deberse a muchos usuarios conectados, red débil o muchas pestañas abiertas. Cierre pestañas innecesarias y verifique su conexión.

\section{Mejores Prácticas}

\subsection{¿Cómo organizar mis transacciones?}

\begin{itemize}
    \item Use fechas correctas y cronológicas
    \item Escriba leyendas claras
    \item Valide regularmente
    \item Genere libros por período
\end{itemize}

\subsection{¿Validar inmediatamente?}

No necesariamente. Puede crear varios y validarlos juntos tras revisarlos. No deje muchas sin validar por largo tiempo.

\subsection{¿Con qué frecuencia genero libros?}

Depende del profesor. Común: un libro al finalizar cada mes o período de práctica.

\section{Soporte}

Para problemas técnicos:
\begin{enumerate}
    \item Consulte este manual
    \item Revise las preguntas frecuentes
    \item Contacte a su profesor o administrador
\end{enumerate}

Si encuentra errores:
\begin{enumerate}
    \item Documente qué estaba haciendo
    \item Note mensajes de error
    \item Reporte al administrador con esta información
\end{enumerate}