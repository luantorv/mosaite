\chapter{Primeros Pasos}

\section{Acceso al Sistema}

\subsection{Inicio de Sesión}

Para acceder a Mosaite, ingrese la URL proporcionada por su institución en su navegador web:

\begin{center}
    \texttt{http://192.168.x.x:3000}
\end{center}

\begin{figure}[H]
    \centering
    \includegraphics[height=10cm, keepaspectratio]{img/login.png}
    \caption{Pantalla de inicio de sesión}
    \label{fig:login}
\end{figure}

Ingrese sus credenciales:
\begin{itemize}
    \item \textbf{Usuario}: Nombre de usuario asignado
    \item \textbf{Contraseña}: Contraseña proporcionada o configurada
\end{itemize}

\subsection{Primer Inicio de Sesión}

Al ingresar por primera vez:
\begin{enumerate}
    \item Verifique sus datos de perfil
    \item Familiarícese con la interfaz principal
    \item Revise las funcionalidades disponibles según su rol
    \item Cambie su contraseña si lo considera necesario
\end{enumerate}

\section{Configuración Inicial del Sistema}

\begin{warningbox}
Esta sección es solo para administradores. Los usuarios estándar pueden omitirla.
\end{warningbox}

El administrador debe configurar el sistema antes del primer uso:

\subsection{Parámetros a Configurar}

\begin{description}
    \item[Modo de Operación] Educativo o Empresarial (define los roles disponibles)
    \item[Formato de Fecha] Cómo se mostrarán las fechas (DD/MM/YYYY, etc.)
    \item[Moneda] Moneda para todos los registros (ARS, USD, EUR, etc.)
    \item[Plan de Cuentas] Revisar y ajustar el plan predeterminado
\end{description}

\begin{figure}[H]
    \centering
    \includegraphics[width=0.8\textwidth]{img/configuracion.png}
    \caption{Pantalla de configuración del sistema}
    \label{fig:config}
\end{figure}

\subsection{Creación de Usuarios}

El administrador debe crear las cuentas para todos los participantes:

\begin{enumerate}
    \item Acceda a \textbf{Usuarios} en el menú lateral
    \item Haga clic en \textbf{Crear Usuario}
    \item Complete los datos requeridos:
    \begin{itemize}
        \item Nombre de usuario
        \item Correo electrónico
        \item Contraseña inicial
        \item Rol asignado
        \item Grupo (en modo educativo)
    \end{itemize}
    \item Guarde el nuevo usuario
\end{enumerate}

\begin{tipbox}
No existe registro público. Todos los usuarios deben ser creados manualmente por el administrador.
\end{tipbox}