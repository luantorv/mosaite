\chapter{Roles y Permisos}

\section{Sistema de Control de Acceso}

Mosaite implementa un sistema basado en roles que determina qué acciones puede realizar cada usuario.

\section{Roles en Modo Educativo}

\subsection{Professor}

Rol administrativo con control total del sistema.

\textbf{Permisos:}
\begin{itemize}
    \item Crear, editar y eliminar usuarios
    \item Configurar el sistema
    \item Administrar el plan de cuentas
    \item Crear, validar y cerrar transacciones
    \item Generar libros diarios
    \item Supervisar trabajo de todos los estudiantes
    \item Acceder al dashboard completo
\end{itemize}

\subsection{Student}

Rol de usuario estándar para práctica contable.

\textbf{Permisos:}
\begin{itemize}
    \item Crear y editar propias transacciones
    \item Validar propias transacciones
    \item Generar libros diarios propios
    \item Consultar el plan de cuentas
    \item Usar Chat RAG y ConsultorIA
    \item Ver propio dashboard
\end{itemize}

\textbf{Restricciones:}
\begin{itemize}
    \item No puede modificar configuración
    \item No puede gestionar usuarios
    \item No puede editar plan de cuentas
    \item No puede ver transacciones de otros
\end{itemize}

\section{Roles en Modo Empresarial}

\begin{table}[H]
\centering
\small
\begin{tabular}{|p{3cm}|p{10cm}|}
\hline
\textbf{Rol} & \textbf{Descripción y Permisos Principales} \\
\hline
\textbf{Admin} & Control total. Todas las operaciones, configuración, gestión de usuarios, administración completa. \\
\hline
\textbf{Manager} & Supervisión amplia. Ver/editar transacciones, generar reportes, gestionar usuarios. No modifica configuración global. \\
\hline
\textbf{Accountant} & Operaciones contables. Crear/validar transacciones, generar libros, administrar cuentas. \\
\hline
\textbf{Operator} & Carga básica. Crear transacciones, consultar plan de cuentas, ver propias transacciones. \\
\hline
\textbf{Viewer} & Solo lectura. Consultar transacciones, ver libros, consultar plan de cuentas, ver dashboards. \\
\hline
\end{tabular}
\caption{Roles en modo empresarial}
\label{tab:roles}
\end{table}

\section{Permisos por Funcionalidad}

\begin{table}[H]
\centering
\small
\begin{tabular}{|p{4cm}|p{2cm}|p{2cm}|p{2cm}|p{2cm}|}
\hline
\textbf{Funcionalidad} & \textbf{Crear} & \textbf{Ver} & \textbf{Editar} & \textbf{Eliminar} \\
\hline
Transacciones & Todos* & Todos & Account+ & Admin \\
\hline
Libros Diarios & Account+ & Todos & Admin & Admin \\
\hline
Usuarios & Admin & Manager+ & Admin & Admin \\
\hline
Plan de Cuentas & Account+ & Account+ & Account+ & Admin \\
\hline
Configuración & Admin & Manager+ & Admin & Admin \\
\hline
\end{tabular}
\caption{Permisos por funcionalidad (*excepto Viewer, +y superiores)}
\label{tab:permisos}
\end{table}

\section{Reglas Especiales}

\subsection{Alcance de Datos}

\begin{description}
    \item[Students] Solo ven sus propias transacciones y libros
    \item[Professors] Ven todo el contenido de sus grupos
    \item[Accountants] Ven sus propios registros
    \item[Managers y Admins] Ven todos los registros
\end{description}

\subsection{Edición de Transacciones}

\begin{itemize}
    \item Solo se editan transacciones "Por Validar"
    \item Una vez validada, solo Admin puede revertir
    \item Transacciones cerradas no se editan nunca
\end{itemize}

\subsection{Eliminación Protegida}

\begin{itemize}
    \item Cuentas usadas en transacciones solo se desactivan
    \item Transacciones en libros cerrados no se eliminan
\end{itemize}

\begin{figure}[H]
    \centering
    \includegraphics[width=0.9\textwidth]{img/gestion_usuarios.png}
    \caption{Pantalla de gestión de usuarios y roles}
    \label{fig:usuarios}
\end{figure}

\section{Verificación de Permisos}

\begin{itemize}
    \item Los menús no disponibles aparecen deshabilitados o no visibles
    \item Intentar una acción sin permiso muestra mensaje de error
    \item Consulte con su administrador si cree que debería tener acceso a alguna funcionalidad
\end{itemize}

\begin{notebox}
Su rol determina completamente qué puede hacer en el sistema. Revise su rol en su perfil de usuario.
\end{notebox}