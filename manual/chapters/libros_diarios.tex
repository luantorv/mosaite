\chapter{Libros Diarios}

\section{¿Qué es un Libro Diario?}

El Libro Diario es un documento contable que registra cronológicamente todas las operaciones de un período determinado. En Mosaite se genera en formato PDF.

\begin{figure}[H]
    \centering
    \includegraphics[width=0.8\textwidth]{img/libro_diario_ejemplo.png}
    \caption{Ejemplo de libro diario generado}
    \label{fig:libro}
\end{figure}

\section{Crear un Libro Diario}

\begin{enumerate}
    \item Acceda a \textbf{Libros Diarios > Crear Libro Diario}
    \item Configure los parámetros:
    \begin{itemize}
        \item \textbf{Nombre}: Identificador descriptivo (ej: "Libro Enero 2025")
        \item \textbf{Rango de fechas}: Período a incluir
        \item \textbf{Filtros}: Por usuario o grupo (opcional)
    \end{itemize}
    \item Haga clic en \textbf{Vista Previa} para revisar transacciones
    \item Verifique la cantidad y totales
    \item Haga clic en \textbf{Generar Libro Diario}
\end{enumerate}

\begin{figure}[H]
    \centering
    \includegraphics[width=0.9\textwidth]{img/crear_libro.png}
    \caption{Formulario de creación de libro diario}
    \label{fig:crear_libro}
\end{figure}

\begin{warningbox}
Al generar un libro, todas las transacciones incluidas quedan en estado "Cerrada" y no pueden modificarse.
\end{warningbox}

\section{Proceso de Generación}

Al generar un libro diario, el sistema:

\begin{enumerate}
    \item Selecciona transacciones \textbf{validadas} del período especificado
    \item Las ordena cronológicamente
    \item Genera un documento PDF estructurado
    \item Cambia el estado de las transacciones a "Cerrada"
    \item Guarda el libro en el sistema
\end{enumerate}

\section{Contenido del Libro Diario}

El PDF generado incluye:

\begin{description}
    \item[Encabezado] Nombre, período, fecha de generación, usuario/grupo
    \item[Cuerpo] Para cada transacción:
        \begin{itemize}
            \item Número de asiento
            \item Fecha
            \item Leyenda
            \item Detalle de cuentas con débitos y créditos
            \item Totales parciales
        \end{itemize}
    \item[Pie] Total general, cantidad de asientos, validación
\end{description}

\section{Buscar y Visualizar Libros}

\subsection{Búsqueda}

Acceda a \textbf{Libros Diarios > Buscar} y use filtros:
\begin{itemize}
    \item Por nombre
    \item Por fecha de creación
    \item Por período que cubren
    \item Por usuario creador
\end{itemize}

\subsection{Visualización}

Para ver un libro diario:
\begin{itemize}
    \item \textbf{Ver en línea}: Visualiza el PDF en el navegador
    \item \textbf{Descargar}: Guarda el PDF en su dispositivo
    \item \textbf{Ver Detalles}: Información del libro y transacciones incluidas
\end{itemize}

\begin{figure}[H]
    \centering
    \includegraphics[width=0.9\textwidth]{img/lista_libros.png}
    \caption{Lista de libros diarios con opciones de visualización}
    \label{fig:lista_libros}
\end{figure}

\section{Eliminación de Libros}

Los administradores pueden eliminar libros diarios:

\begin{enumerate}
    \item Acceda al libro que desea eliminar
    \item Seleccione \textbf{Eliminar}
    \item Confirme la acción
\end{enumerate}

\begin{notebox}
Al eliminar un libro, las transacciones vuelven al estado "Validada" y pueden incluirse en un nuevo libro.
\end{notebox}

\section{Recomendaciones}

\begin{itemize}
    \item Genere libros periódicamente (mensual, bimestral)
    \item Use nombres descriptivos con el período
    \item Descargue y respalde los PDF generados
    \item En modo educativo, considere libros individuales por estudiante
    \item Revise la vista previa antes de generar
\end{itemize}

\begin{tipbox}
Es buena práctica generar un libro diario al finalizar cada período de práctica para mantener registros organizados.
\end{tipbox}